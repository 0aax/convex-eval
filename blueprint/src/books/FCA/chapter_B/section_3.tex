\subsection{Continuity properties}
\begin{lemma}
\label{lem:3.1.1}
Let $f\in\operatorname{Conv}\mathbb{R}^n$ and suppose there are $x_0,\ \delta,\ m$ and $M$ such that
\[
m\le f(x)\le M\quad\text{for all }x\in B(x_0,2\delta).
\]
Then $f$ is Lipschitzian on $B(x_0,\delta)$; more precisely: for all $y$ and $y'$ in $B(x_0,\delta)$,
\begin{equation}
\label{eq:3.1.1}
|f(y)-f(y')|\le\frac{M-m}{\delta}\|y-y'\|.
\end{equation}
\end{lemma}

\begin{proof}
Look at Fig.\ 3.1.1: with two different $y$ and $y'$ in $B(x_0,\delta)$, take
\[
y'':=y'+\delta\,\frac{y'-y}{\|y'-y\|}\in B(x_0,2\delta);
\]

by construction, $y'$ lies on the segment $[y,y'']$, namely
\[
y'=\frac{\|y'-y\|}{\delta+\|y'-y\|}\,y''+\frac{\delta}{\delta+\|y'-y\|}\,y.
\]
Applying the convexity of $f$ and using the postulated bounds, we obtain
\[
f(y')-f(y)\le\frac{\|y'-y\|}{\delta+\|y'-y\|}\bigl[f(y'')-f(y)\bigr]\le\frac{1}{\delta}\|y'-y\|(M-m).
\]
Then, it suffices to exchange $y$ and $y'$ to prove (3.1.1).
\end{proof}

\begin{theorem}\label{thm:3.1.2}
With $f\in\operatorname{Conv}\mathbb{R}^n$, let $S$ be a convex compact subset of \(\operatorname{ri}\dom f\). Then there exists \(L=L(S)\ge 0\) such that
\[
|f(x)-f(x')|\le L\|x-x'\|\quad\text{for all }x\text{ and }x'\text{ in }S.
\tag{3.1.2}
\]
\end{theorem}

\begin{proof}
[Preliminaries] First of all, our statement ignores \(x\)-values outside the affine hull of the convex set \(\dom f\). Instead of \(\mathbb{R}^n\), it can be formulated in \(\mathbb{R}^d\), where \(d\) is the dimension of \(\dom f\); alternatively, we may assume \(\ri\dom f=\int\dom f\), which will simplify the writing.

Make this assumption and let \(x_0\in S\). We will prove that there are \(\delta=\delta(x_0)>0\) and \(L=L(x_0,\delta)\) such that the ball \(B(x_0,\delta)\) is included in \(\int\dom f\) and
\[
|f(y)-f(y')|\le L\|y-y'\|\quad\text{for all }y\text{ and }y'\text{ in }B(x_0,\delta).
\tag{3.1.3}
\]

If this holds for all \(x_0\in S\), the corresponding balls \(B(x_0,\delta)\) will provide a covering of the compact set \(S\), from which we will extract a finite covering \((x_1,\delta_1,L_1),\ldots,(x_k,\delta_k,L_k)\). With these balls, we will divide an arbitrary segment \([x,x']\) of the convex set \(S\) into finitely many subsegments, of endpoints \(y_0:=x,\ldots,y_i,\ldots,y_\ell:=x'\). Ordering properly the \(y_i\)'s, we will have \(\|x-x'\|=\sum_{i=1}^\ell\|y_i-y_{i-1}\|\); furthermore, \(f\) will be Lipschitzian on each \([y_{i-1},y_i]\) with the common constant \(L:=\max\{L_1,\ldots,L_k\}\). The required Lipschitz property (3.1.2) will follow.

[Main Step] To establish (3.1.3), we use Lemma 3.1.1, which requires boundedness of $f$ in the neighborhood of $x_0$. For this, we construct as in the proof of Theorem A.2.1.3 (see Fig.\ A.2.1.1) a simplex $\Delta=\operatorname{co}\{v_0,\dots,v_n\}\cap\operatorname{dom}f$ having $x_0$ in its interior: we can take $\delta>0$ such that $B(x_0,2\delta)\subset\Delta$.

Then any $y\in B(x_0,2\delta)$ can be written: $y=\sum_{i=0}^n\alpha_i v_i$ with $\alpha\in\Delta_{n+1}$, so that the convexity of $f$ gives
\[
f(y)\le\sum_{i=0}^n\alpha_i f(v_i)\le\max\{f(v_0),\dots,f(v_n)\} =: M .
\]

On the other hand, Proposition 1.2.1 tells us that $f$ is bounded from below, say by $m$, on this very same $B(x_0,2\delta)$. Our claim is proved: we have singled out $\delta>0$ such that $m\le f(y)\le M$ for all $y\in B(x_0,2\delta)$.
\end{proof}

\begin{theorem}\label{thm:3.1.4}
Let the convex functions $f_k:\mathbb{R}^n\to\mathbb{R}$ converge pointwise for $k\to+\infty$ to $f:\mathbb{R}^n\to\mathbb{R}$. Then $f$ is convex and, for each compact set $S$, the convergence of $f_k$ to $f$ is uniform on $S$.
\end{theorem}

\begin{proof}
Convexity of $f$ is trivial: pass to the limit in the definition (1.1.1) itself. For uniformity, we want to use Lemma 3.1.1, so we need to bound $f_k$ on $S$ independently of $k$; thus, let $r>0$ be such that $S\subset B(0,r)$.

[Step 1] First the function $g:=\sup_k f_k$ is convex, and $g(x)<+\infty$ for all $x$ because the convergent sequence $(f_k(x))_k$ is certainly bounded. Hence, $g$ is continuous and therefore bounded, say by $M$, on the compact set $B(0,2r)$:
\[
f_k(x)\le g(x)\le M\qquad\text{for all $k$ and all $x\in B(0,2r)$.}
\]

Second, the convergent sequence $(f_k(0))_k$ is bounded from below:
\[
\mu\le f_k(0)\qquad\text{for all $k$.}
\]

Then, for $x\in B(0,2r)$ and all $k$, use convexity on $[-x,x]\subset B(0,2r)$:
\[
2\mu\le 2f_k(0)\le f_k(x)+f_k(-x)\le f_k(x)+M,
\]
i.e.\ the $f_k$'s are bounded from below, independently of $k$. Thus, we are within the conditions of Lemma 3.1.1: there is some $L$ (independent of $k$) such that
\[
|f_k(y)-f_k(y')| \le L\|y-y'\| \quad\text{for all }k\text{ and all }y,y'\in B(0,r). \tag{3.1.5}
\]

Naturally, the same Lipschitz property is transmitted to the limiting function $f$.

[Step 2] Now fix $\varepsilon>0$. Cover $S$ by the balls $B(x,\varepsilon)$ for $x$ describing $S$, and extract a finite covering $S\subset B(x_1,\varepsilon)\cup\cdots\cup B(x_m,\varepsilon)$. With $x$ arbitrary in $S$, take an $x_i$ such that $x\in B(x_i,\varepsilon)$. There is $k_{i,\varepsilon}$ such that, for all $k\ge k_{i,\varepsilon}$,
\[
|f_k(x)-f(x)| \le |f_k(x)-f_k(x_i)|+|f_k(x_i)-f(x_i)|+|f(x_i)-f(x)| \le (2L+1)\varepsilon
\]
where we have also used (3.1.5), knowing that $x$ and $x_i$ are in $S\subset B(0,r)$. The above inequality is then valid uniformly in $x$, providing that
\[
k\ge \max\{k_{1,\varepsilon},\ldots,k_{m,\varepsilon}\} =: k_\varepsilon.
\]
\end{proof}

\subsection{Behavior at infinity}

\begin{proposition}
\label{prop:3.2.1}
For $f\in\overline{\operatorname{Conv}}\mathbb{R}^n$, the asymptotic cone of $\operatorname{epi} f$ is the epigraph of the function $f'_{\infty}\in\overline{\operatorname{Conv}}\mathbb{R}^n$ defined by
\[
d\mapsto f'_{\infty}(d):=\sup_{t>0}\frac{f(x_0+td)-f(x_0)}{t}
\;=\;
\lim_{t\to+\infty}\frac{f(x_0+td)-f(x_0)}{t},
\tag{3.2.2}
\]
where $x_0$ is arbitrary in $\operatorname{dom} f$.
\end{proposition}

\begin{proof}
Since $(x_0,f(x_0))\in\operatorname{epi} f$, (3.2.1) tells us that $(d,\rho)\in(\operatorname{epi} f)_{\infty}$ if and only if $f(x_0+td)\le f(x_0)+t\rho$ for all $t>0$, which means

\[
\sup_{t>0}\frac{f(x_0+td)-f(x_0)}{t}\le \rho. \tag{3.2.3}
\]

In other words, $(\operatorname{epi} f)_\infty$ is the epigraph of the function whose value at $d$ is the left hand side of (3.2.3); and this is true no matter how $x_0$ has been chosen in \(\operatorname{dom} f\). The rest follows from the fact that the difference quotient in (3.2.3) is closed convex in \(d\), and increasing in \(t\) (the function \(t\mapsto f(x_0+td)\) is convex and enjoys the property of increasing slopes, namely Proposition 0.6.1). \(\square\)
\end{proof}

\begin{definition}
\label{def:3.2.2}
(Asymptotic function) The function $f'_{\infty}$ of Proposition 3.2.1 is called the \emph{asymptotic function}, or recession function, of $f$.
\end{definition}

\begin{proposition}
Let $f\in\operatorname{Conv}\mathbb{R}^n$. All the nonempty sublevel-sets of $f$ have the same asymptotic cone, which is the sublevel-set of $f^\infty$ at the level $0$:
\[
\forall r\in\mathbb{R}\ \text{with } S_r(f)\neq\varnothing,\qquad [S_r(f)]_\infty=\{d\in\mathbb{R}^n:\ f^\infty(d)\le 0\}.
\]
In particular, the following statements are equivalent:
\begin{enumerate}
    \item[(i)] There is $r$ for which $S_r(f)$ is nonempty and compact;
    \item[(ii)] all the sublevel-sets of $f$ are compact;
    \item[(iii)]  $f^\infty_\circ(d)>0$ for all nonzero $d\in\mathbb{R}^n$.
\end{enumerate}
\end{proposition}

\begin{proof}
By definition (A.2.2.1), a direction $d$ is in the asymptotic cone of the nonempty sublevel-set $S_r(f)$ if and only if
\[
x\in S_r(f)\qquad\Longrightarrow\qquad [x+td\in S_r(f)\ \text{for all }t>0],
\]
which can also be written --- see (1.1.4):
\[
(x,r)\in\operatorname{epi} f\qquad\Longrightarrow\qquad (x+td,r+t\times 0)\in\operatorname{epi} f\ \text{ for all }t>0;
\]
and this in turn just means that $(d,0)\in(\operatorname{epi} f)_\infty=\operatorname{epi} f^\infty_\circ$. We have proved the first part of the theorem.

A particular case is when the sublevel-set $S_0(f^\infty_\circ)$ is reduced to the singleton $\{0\}$, which exactly means (iii). This is therefore equivalent to $[S_r(f)]_\infty=\{0\}$ for all $r\in\mathbb{R}$ with $S_r(f)\neq\emptyset$, which means that $S_r(f)$ is compact (Proposition A.2.2.3). The equivalence between (i), (ii) and (iii) is proved.
\end{proof}

\begin{definition}
\label{def:3.2.5}
(Coercivity) The functions $f\in\operatorname{Conv}\mathbb{R}^n$ satisfying (i), (ii) or (iii) are called $0$-coercive. Equivalently, the $0$-coercive functions are those that ``increase at infinity'':
\[
f(x)\to +\infty\qquad\text{whenever}\qquad \|x\|\to +\infty,
\]
and closed convex $0$-coercive functions achieve their minimum over $\mathbb{R}^n$.

An important particular case is when $f'_\infty(d)=+\infty$ for all $d\neq 0$, i.e.\ when $f'_\infty = \iota_{\{0\}}$. It can be seen that this means precisely
\[
\frac{f(x)}{\|x\|}\to +\infty\qquad\text{whenever}\qquad \|x\|\to +\infty.
\]

(to establish this equivalence, extract a cluster point of $(x_k/\|x_k\|)$, and use the lower semi-continuity of $f'_\infty$). In words: at infinity, $f$ increases to infinity faster than any affine function; such functions are called $1$-coercive, or sometimes just coercive.
\end{definition}

\begin{proposition}
\label{prop:3.2.6}
A function $f\in\operatorname{Conv}\mathbb{R}^n$ is Lipschitzian on the whole of $\mathbb{R}^n$ if and only if $f'_\infty$ is finite on the whole of $\mathbb{R}^n$. The best Lipschitz constant for $f$ is then
\[
\sup\{f'_\infty(d):\|d\|=1\}.
\tag{3.2.4}
\]
\end{proposition}

\begin{proof}
When the (convex) function $f'_\infty$ is finite-valued, it is continuous (\S3.1) and therefore bounded on the compact unit sphere:
\[
\sup\{f'_\infty(d):\|d\|=1\} =: L < +\infty,
\]
which implies by positive homogeneity
\[
f'_\infty(d) \le L\|d\|\quad\text{for all }d\in\mathbb{R}^n.
\]
Now use the definition (3.2.2) of $f'_\infty$:
\[
f(x+d)-f(x)\le L\|d\|\quad\text{for all }x\in\operatorname{dom}f\text{ and }d\in\mathbb{R}^n;
\]
thus, $\operatorname{dom}f$ is the whole space ($f(x+d)<+\infty$ for all $d$) and we do obtain that $L$ is a global Lipschitz constant for $f$.

Conversely, let $f$ have a global Lipschitz constant $L$. Pick $x_0\in\operatorname{dom}f$ and plug the inequality
\[
f(x_0+td)-f(x_0)\le Lt\|d\|\quad\text{for all }t>0\text{ and }d\in\mathbb{R}^n
\]
into the definition (3.2.2) of $f'_\infty$ to obtain $f'_\infty(d)\le L\|d\|$ for all $d\in\mathbb{R}^n$.

It follows that $f'_\infty$ is finite everywhere, and the value (3.2.4) does not exceed $L$.
\end{proof}

\begin{proposition}[3.2.8]
\label{thm:3.2.8}
\begin{enumerate}
\item Let $f_1,\dots,f_m$ be $m$ functions of $\Conv\R^n$, and $t_1,\dots,t_m$ be positive numbers. Assume that there is $x_0$ at which each $f_j$ is finite. Then,
\[
\text{for } f:=\sum_{j=1}^m t_j f_j,\qquad\text{we have } f'_\infty=\sum_{j=1}^m t_j(f_j)'_\infty.
\]

\item Let $\{f_j\}_{j\in J}$ be a family of functions in $\Conv\R^n$. Assume that there is $x_0$ at which $\sup_{j\in J} f_j(x_0)<+\infty$. Then,
\[
\text{for } f:=\sup_{j\in J} f_j,\qquad\text{we have } f'_\infty=\sup_{j\in J}(f_j)'_\infty.
\]

\item Let $A:\R^n\to\R^m$ be affine with linear part $A_0$, and let $f\in\Conv\R^m$. Assume that $A(\R^n)\cap\dom f\neq\emptyset$. Then $(f\circ A)'_\infty=f'_\infty\circ A_0$.
\end{enumerate}
\end{proposition}