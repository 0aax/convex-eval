\subsection{The definitions of a Convex Function}

\begin{definition}
\label{def:1.1.1}
Let $C$ be a nonempty convex set in $\mathbb{R}^n$.  A function $f : C \to \mathbb{R}$ is said to be \emph{convex on $C$} when, for all pairs $(x,x')\in C\times C$ and all $\alpha\in[0,1]$, there holds
\[
f(\alpha x + (1-\alpha)x') \le \alpha f(x) + (1-\alpha) f(x').
\tag{1.1.1}
\]
\end{definition}

\begin{proposition}
\label{prop:1.1.2}
\lean{FCA_chap_B_1_1_2}
The function $f$ is strongly convex on $C$ with modulus $c$ if and only if the function $f-\tfrac{1}{2}c\|\cdot\|^{2}$ is convex on $C$.
\end{proposition}

\begin{proof}
Use direct calculations in the definition (1.1.1) of convexity applied to the function $f-\tfrac{1}{2}c\|\cdot\|^{2}$, namely:
\[
f(\alpha x+(1-\alpha)x')-\tfrac{1}{2}c\|\alpha x+(1-\alpha)x'\|^{2}\le
\]
\[
\le \alpha f(x)+(1-\alpha)f(x')-\tfrac{1}{2}c\big[\alpha\|x\|^{2}+(1-\alpha)\|x'\|^{2}\big].
\]
\end{proof}

\begin{definition}
\label{def:1.1.3}
(The Set $\operatorname{Conv}\mathbb{R}^n$) A function $f:\mathbb{R}^n\to\mathbb{R}\cup\{+\infty\}$, not identically $+\infty$, is said to be convex when, for all $(x,x')\in\mathbb{R}^n\times\mathbb{R}^n$ and all $\alpha\in[0,1]$, there holds
\[
f\big(\alpha x+(1-\alpha)x'\big)\le \alpha f(x)+(1-\alpha)f(x'),
\]
considered as an inequality in $\mathbb{R}\cup\{+\infty\}$. 

The class of such functions is denoted by $\operatorname{Conv}\mathbb{R}^n$.
\end{definition}

\begin{definition}
\label{def:1.1.4}
(Domain of a Function) The \emph{domain} (or also effective domain) of $f\in\Conv\mathbb{R}^n$ is the nonempty set
\[
\dom f := \{x\in\mathbb{R}^n : f(x) < +\infty\}.
\]
\end{definition}

\begin{definition}
\label{def:1.1.5}
(Epigraph of a Function) Given $f:\mathbb{R}^n\to\mathbb{R}\cup\{+\infty\}$, not identically equal to $+\infty$, the \emph{epigraph} of $f$ is the nonempty set
\[
\operatorname{epi} f := \{(x,r)\in\mathbb{R}^n\times\mathbb{R} : r \ge f(x)\}.
\]
Its \emph{strict epigraph} $\operatorname{epi}_s f$ is defined likewise, with ``$\ge$'' replaced by ``$>$'' (beware that the word ``strict'' here has nothing to do with strict convexity).
\end{definition}

\begin{proposition}
\label{prop:1.1.6}
\lean{FCA_chap_B_1_1_6}
Let $f:\mathbb{R}^n\to\mathbb{R}\cup\{+\infty\}$ be not identically equal to $+\infty$. The three properties below are equivalent:
\begin{enumerate}
\item[(i)] $f$ is convex in the sense of Definition 1.1.3;
\item[(ii)] its epigraph is a convex set in $\mathbb{R}^n\times\mathbb{R}$;
\item[(iii)] its strict epigraph is a convex set in $\mathbb{R}^n\times\mathbb{R}$.
\end{enumerate}
\end{proposition}

\begin{proof}
Left as an exercise.
\end{proof}

\begin{theorem}
\label{thm:1.1.8}
\lean{FCA_chap_B_1_1_8}
(Inequality of Jensen) Let $f\in\operatorname{Conv}\mathbb{R}^n$. Then, for all collections $\{x_1,\dots,x_k\}$ of points in $\dom f$ and all $\alpha=(\alpha_1,\dots,\alpha_k)$ in the unit simplex of $\mathbb{R}^k$, there holds (inequality of Jensen in summation form)
\[
f\!\bigg(\sum_{i=1}^k \alpha_i x_i\bigg)\le \sum_{i=1}^k \alpha_i f(x_i).
\]
\end{theorem}

\begin{proof}
The $k$ points $(x_i,f(x_i))\in\mathbb{R}^n\times\mathbb{R}$ are clearly in $\epi f$, a convex set. Their convex combination
\[
\sum_i \alpha_i (x_i,f(x_i)) = \Big(\sum_i \alpha_i x_i,\sum_i \alpha_i f(x_i)\Big)
\]
is also in $\epi f$ (Proposition A.1.3.3). This is just the claimed inequality.
\end{proof}

\begin{proposition}
\label{prop:1.1.9}
\lean{FCA_chap_B_1_1_9}
Let $f\in\Conv\R^n$.  The relative interior of $\operatorname{epi}f$ is the union over $x\in\operatorname{ri}\,\operatorname{dom}f$ of the open half-lines with bottom endpoints at $f(x)$:
\[
\operatorname{ri}\operatorname{epi}f=\{(x,r)\in\R^n\times\R : x\in\operatorname{ri}\operatorname{dom}f,\ r>f(x)\}.
\]
\end{proposition}

\begin{proof}
Since $\operatorname{dom}f$ is the image of $\operatorname{epi}f$ under the linear mapping ``projection onto $\R^n$'', Proposition A.2.1.12 tells us that
\[
\operatorname{ri}\operatorname{dom}f\ \text{is the projection onto }\R^n\ \text{of }\operatorname{ri}\operatorname{epi}f.
\tag{1.1.5}
\]
Now take $x$ arbitrary in $\operatorname{ri}\operatorname{dom}f$.  The subset of $\operatorname{ri}\operatorname{epi}f$ that is projected onto $x$ is just $((\{x\}\times\R)\cap\operatorname{ri}\operatorname{epi}f)$, which in turn is $\operatorname{ri}((\{x\}\times\R)\cap\operatorname{epi}f)$ (use Proposition A.2.1.10).  This latter set is clearly $\{x\}\times(f(x),+\infty)$.

In summary, we have proved that, for $x\in\operatorname{ri}\operatorname{dom}f$, $(x,r)\in\operatorname{ri}\operatorname{epi}f$ if and only if $r>f(x)$.  Together with (1.1.5), this proves our claim.
\end{proof}

\subsection{Special convex functions: Affinity and Closedness}

\begin{proposition}
\label{thm:1.2.1}
\lean{FCA_chap_B_1_2_1}
Any $f\in\operatorname{Conv}\mathbb{R}^n$ is minorized by some affine function.  More precisely: for any $x_0\in\operatorname{ri}\operatorname{dom}f$, there is $s$ in the subspace parallel to $\operatorname{aff}\operatorname{dom}f$ such that
\[
f(x)\ge f(x_0)+\langle s,\,x-x_0\rangle\quad\text{for all }x\in\mathbb{R}^n.
\]
\end{proposition}

In other words, the affine function can be forced to coincide with $f$ at $x_0$.

\begin{proof}
We know that $\operatorname{dom}f$ is the image of $\operatorname{epi}f$ under the linear mapping "projection onto $\mathbb{R}^n$".  Look again at the definition of an affine hull (\S A.1.3) to realize that
\[
\operatorname{aff}\operatorname{epi}f=(\operatorname{aff}\operatorname{dom}f)\times\mathbb{R}\,.
\]

Denote by $V$ the linear subspace parallel to $\operatorname{aff}\operatorname{dom}f$, so that $\operatorname{aff}\operatorname{dom}f=\{x_0\}+V$ with $x_0$ arbitrary in $\operatorname{dom}f$; then we have
\begin{equation}\label{eq:1.2.1}
\operatorname{aff}\operatorname{epi}f=\{x_0+V\}\times\mathbb{R}.
\end{equation}

We equip $V\times\mathbb{R}$ and $\mathbb{R}^n\times\mathbb{R}$ with the scalar product of product-spaces.  With $x_0\in\operatorname{ri}\operatorname{dom}f$, Proposition 1.1.9 tells us that $(x_0,f(x_0))\in\operatorname{rbd}\operatorname{epi}f$ and we can take a nontrivial hyperplane supporting $\operatorname{epi}f$ at $(x_0,f(x_0))$: using Remark A.4.2.2 and (\ref{eq:1.2.1}), there are $s=s_V\in V$ and $\alpha\in\mathbb{R}$, not both zero, such that
\begin{equation}\label{eq:1.2.2}
\langle s,x\rangle+\alpha r\le\langle s,x_0\rangle+\alpha f(x_0)
\end{equation}
for all $(x,r)$ with $f(x)\le r$.  Note: this implies $\alpha\le 0$ (let $r\to+\infty$!).

Because of our choice of $s$ (in $V$) and $x_0$ (in $\operatorname{ri}\operatorname{dom}f$), we can take $\delta>0$ so small that $x_0+\delta s\in\operatorname{dom}f$, for which (\ref{eq:1.2.2}) gives
\[
\delta\|s\|^2\le\alpha\bigl[f(x_0)-f(x_0+\delta s)\bigr]<+\infty;
\]
this shows $\alpha\neq 0$ (otherwise, both $s$ and $\alpha$ would be zero).  Without loss of generality, we can assume $\alpha=-1$; then (\ref{eq:1.2.2}) gives our affine function.
\end{proof}

\begin{proposition}
\label{prop:1.2.2}
\lean{FCA_chap_B_1_2_2}
For $f:\mathbb{R}^n\to\mathbb{R}\cup\{+\infty\}$, the following three properties are equivalent:
\begin{enumerate}
\item[(i)] $f$ is lower semi-continuous on $\mathbb{R}^n$;
\item[(ii)] $\operatorname{epi} f$ is a closed set in $\mathbb{R}^n\times\mathbb{R}$;
\item[(iii)] the sublevel-sets $S_r(f)$ are closed (possibly empty) for all $r\in\mathbb{R}$.
\end{enumerate}
\end{proposition}

\begin{proof}

[(i) $\Rightarrow$ (ii)] Let $(y_k,r_k)_k$ be a sequence of $\operatorname{epi} f$ converging to $(x,r)$ for $k\to+\infty$. Since $f(y_k)\le r_k$ for all $k$, the l.s.c.\ relation (1.2.3) readily gives
\[
r=\lim r_k \ge \liminf f(y_k)\ge \liminf_{y\to x} f(y)\ge f(x),
\]
i.e.\ $(x,r)\in\operatorname{epi} f$.

[(ii) $\Rightarrow$ (iii)] Construct the sublevel-sets $S_r(f)$ as in Remark 1.1.7: the closed sets $\operatorname{epi} f$ and $\mathbb{R}^n\times\{r\}$ have a closed intersection.

[(iii) $\Rightarrow$ (i)] Suppose $f$ is not lower semi-continuous at some $x$: there is a (sub)sequence $(y_k)$ converging to $x$ such that $f(y_k)$ converges to $\rho<f(x)\le +\infty$. Pick $r\in[\rho,f(x))$: for $k$ large enough, $f(y_k)\le r< f(x)$; hence $S_r(f)$ contains the tail of $(y_k)$ but not its limit $x$. Consequently, this $S_r(f)$ is not closed.

\end{proof}

\begin{definition}
\label{def:1.2.3}
(Closed Functions) The function $f:\mathbb{R}^n\to\mathbb{R}\cup\{+\infty\}$ is said to be closed if it is lower semi-continuous everywhere, or if its epigraph is closed, or if its sublevel-sets are closed.
\end{definition}

\begin{definition}[Closure of a Function]\label{def:1.2.4}
The \emph{closure} (or lower semi-continuous hull) of a function $f$ is the function $\operatorname{cl}f:\mathbb{R}^n\to\mathbb{R}\cup\{\pm\infty\}$ defined by:
\[
\operatorname{cl}f(x):=\liminf_{y\to x} f(y)\qquad\text{for all }x\in\mathbb{R}^n,
\tag{1.2.4}
\]
or equivalently by
\[
\operatorname{epi}(\operatorname{cl}f):=\operatorname{cl}\big(\operatorname{epi} f\big).
\tag{1.2.5}
\]
\end{definition}

\begin{proposition}
\label{prop:1.2.5}
\lean{FCA_chap_B_1_2_5}
Let $f\in\operatorname{Conv}\mathbb{R}^n$ and $x'\in\operatorname{ri}\operatorname{dom}f$. There holds (in $\mathbb{R}\cup\{+\infty\}$)
\[
\operatorname{cl}f(x)=\lim_{t\downarrow 0}f\big(x+t(x'-x)\big)\quad\text{for all }x\in\mathbb{R}^n.
\]
\end{proposition}

\begin{proof}
Since $x_t:=x+t(x'-x)\to x$ when $t\downarrow 0$, we certainly have
\[
(\operatorname{cl}f)(x)\le\liminf_{t\downarrow 0}f\big(x+t(x'-x)\big).
\]

We will prove the converse inequality by showing that
\[
\limsup_{t\downarrow 0}f\big(x+t(x'-x)\big)\le r\quad\text{for all }r\ge(\operatorname{cl}f)(x).
\]

(Non-existence of such an $r$ means that $\operatorname{cl}f(x)=+\infty$, the proof is finished.)

Thus let $(x,r)\in\operatorname{epi}(\operatorname{cl}f)=\operatorname{cl}(\operatorname{epi}f)$. Pick $r'>f(x')$, hence $(x',r')\in\operatorname{epi}f$ (Proposition 1.1.9). Applying Lemma A.2.1.6 to the convex set $\operatorname{epi}f$, we see that
\[
t(x',r')+(1-t)(x,r)\in\operatorname{epi}f\subset\operatorname{epi}f\quad\text{for all }t\in]0,1].
\]

This just means
\[
f\big(x+t(x'-x)\big)\le tr'+(1-t)r\quad\text{for all }t\in]0,1]
\]
and our required inequality follows by letting $t\downarrow 0$.
\end{proof}

\begin{proposition}
\label{prop:1.2.6}
\lean{FCA_chap_B_1_2_6}
For $f\in\Conv\R^n$, there holds
\[
\operatorname{cl} f\in\Conv\R^n;
\tag{1.2.7}
\]
\[
\operatorname{cl} f\text{ and }f\text{ coincide on the relative interior of }\dom f.
\tag{1.2.8}
\]
\end{proposition}

\begin{proof}
We already know from Proposition A.1.2.6 that $\epi\operatorname{cl} f=\operatorname{cl}\epi f$ is a convex set; also $\operatorname{cl} f\le f\not\equiv+\infty$; finally, Proposition 1.2.1 guarantees in the relation of definition (1.2.4) that $\operatorname{cl} f(x)>-\infty$ for all $x$: (1.2.7) does hold.

On the other hand, suppose $x\in\ri\dom f$. Then the one-dimensional function $\varphi(t)=f(x+td)$ is continuous at $t=0$ (Theorem 0.6.2); it follows from Proposition 1.2.5 that $\operatorname{cl} f$ coincides with $f$ on $\ri\dom f$; besides, $\operatorname{cl} f(x)$ is obviously equal to $f(x)=+\infty$ for all $x\notin\cl\dom f$. Altogether, (1.2.8) is true.
\end{proof}

Notation 1.2.7 (The Set $\overline{\operatorname{Conv}}\mathbb{R}^n$) The set of closed convex functions on $\mathbb{R}^n$ is denoted by $\overline{\operatorname{Conv}}\mathbb{R}^n$. \qed

\begin{proposition}
\label{thm:1.2.8}
\lean{FCA_chap_B_1_2_8}
The closure of $f\in\Conv\R^n$ is the supremum of all affine functions minorizing $f$:
\[
\cl f(x)=\sup_{(s,b)\in\R^n\times\R}\{\langle s,x\rangle-b:\langle s,y\rangle-b\le f(y)\text{ for all }y\in\R^n\}.
\tag{1.2.9}
\]
\end{proposition}

\begin{proof}
A closed half-space containing $\epi f$ is characterized by a nonzero vector $(s,\alpha)\in\R^n\times\R$ and a real number $b$ such that
\[
\langle s,x\rangle+\alpha r\le b\qquad\text{for all }(x,r)\in\epi f
\tag{1.2.10}
\]
(we equip the graph-space $\R^n\times\R$ with the scalar product of a product-space). Let us denote by $\Sigma\subset\R^n\times\R\times\R$ the index-set of such triples $\sigma=(s,\alpha,b)$, with corresponding half-space
\[
H^-_{\sigma}:=\{(x,r):\langle s,x\rangle+\alpha r\le b\}.
\tag{1.2.11}
\]

In other words, $\operatorname{epi}(\overline{cl}\,f)=\overline{\operatorname{epi}f}=\bigcap_{\sigma\in\Sigma}H^-_{\sigma}$.

Because of the particular nature of an epigraph, (1.2.10) implies $\alpha\le 0$ (let $r\to +\infty$) and, by positive homogeneity, the values $\alpha=0$ and $\alpha=-1$ suffice: $\Sigma$ can be partitioned in

\[
\Sigma_1 := \{(s,-1,b):(1.2.10)\ \text{holds with }\alpha=-1\}
\]

and

\[
\Sigma_0 := \{(s,0,b):(1.2.10)\ \text{holds with }\alpha=0\}.
\]

Indeed, $\Sigma_1$ corresponds to affine functions minorizing $f$ (Proposition 1.2.1 tells us that $\Sigma_1\neq\varnothing$) and $\Sigma_0$ to closed half-spaces of $\mathbb{R}^{n}$ containing $\operatorname{dom}f$ (note that $\Sigma_0=\varnothing$ if $\operatorname{dom}f=\mathbb{R}^n$).

We have to prove that, even when $\Sigma_0\neq\varnothing$, intersecting the half-spaces $H^-_{\sigma}$ over $\Sigma$ or over $\Sigma_1$ produces the same set, namely $\operatorname{epi}f$. For this we take arbitrary $\sigma_0=(s_0,0,b_0)\in\Sigma_0$ and $\sigma_1=(s_1,-1,b_1)\in\Sigma_1$, we set

\[
\sigma(t):=(s_1+ts_0,-1,b_1+t b_0)\in\Sigma_1\quad\text{for all }t\ge 0,
\]

and we prove (see Fig. 1.2.1)

\[
H^-_{\sigma_0}\cap H^-_{\sigma_1}=\bigcap_{t\ge 0}H^-_{\sigma(t)}=:\,H^-.
\]

Fig. 1.2.1. Closing a convex epigraph

It results directly from the definition (1.2.11) that an $(x,r)$ in $H^-_{\sigma_0}\cap H^-_{\sigma_1}$ satisfies

\[
\langle s_1+ts_0,x\rangle-(b_1+tb_0)\le r\quad\text{for all }t\ge 0,\tag{1.2.12}
\]

i.e.\ $(x,r)\in H^-$. Conversely, take $(x,r)\in H^-$. Set $t=0$ in (1.2.12) to see that $(x,r)\in H^-_{\sigma_1}$. Also, divide by $t>0$ and let $t\to+\infty$ to see that $(x,r)\in H^-_{\sigma_0}$. The proof is complete.
\end{proof}

\subsection{First examples}
\begin{theorem}
\label{thm:1.3.1}
\lean{FCA_chap_B_1_3_1}
Let $C$ be a nonempty subset of $\mathbb{R}^n\times\mathbb{R}$ satisfying \eqref{1.3.2}, and let its lower-bound function $\ell_C$ be defined by \eqref{1.3.5}.
\begin{enumerate}[(i)]
\item If $C$ is convex, then $\ell_C\in\operatorname{Conv}\mathbb{R}^n$;
\item If $C$ is closed convex, then $\ell_C\in\overline{\operatorname{Conv}}\mathbb{R}^n$.
\end{enumerate}
\end{theorem}

\begin{proof}
We use the analytical definition (1.1.1). Take arbitrary $\varepsilon>0$, $\alpha\in]0,1[$ and $(x_i,r_i)\in C$ such that $r_i\le \ell_C(x_i)+\varepsilon$ for $i=1,2$.

When $C$ is convex, $(\alpha x_1+(1-\alpha)x_2,\alpha r_1+(1-\alpha)r_2)\in C$, hence
\[
\ell_C(\alpha x_1+(1-\alpha)x_2)\le \alpha r_1+(1-\alpha)r_2 \le \alpha\ell_C(x_1)+(1-\alpha)\ell_C(x_2)+\varepsilon.
\]

The convexity of $\ell_C$ follows, since $\varepsilon>0$ was arbitrary; (i) is proved.

Now take a sequence $(x_k,\rho_k)_k\subset\operatorname{epi}\ell_C$ converging to $(x,\rho)$; we have to prove $\ell_C(x)\le \rho$ (cf.\ Proposition 1.2.2). By definition of $\ell_C(x_k)$, we can select, for each positive integer $k$, a real number $r_k$ such that $(x_k,r_k)\in C$ and
\begin{equation}\label{eq:1.3.7}
\ell_C(x_k)\le r_k\le \ell_C(x_k)+\tfrac{1}{k}\le \rho_k+\tfrac{1}{k}.
\end{equation}

We deduce first that $(r_k)$ is bounded from above. Also, when $\ell_C$ is convex, Proposition 1.2.1 implies the existence of an affine function minorizing $\ell_C$: $(r_k)$ is bounded from below.

Extracting a subsequence if necessary, we may assume $r_k\to r$. When $C$ is closed, $(x,r)\in C$, hence $\ell_C(x)\le r$; but pass to the limit in \eqref{eq:1.3.7} to see that $r\le \rho$; the proof is complete.
\end{proof}