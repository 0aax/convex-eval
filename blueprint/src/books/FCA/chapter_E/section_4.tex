\subsection{First-order differentiability}

\begin{theorem}\label{thm:FCA-chapE-4.1.1}
Let $f\in\operatorname{Conv}\R^n$ be strictly convex. Then $\operatorname{int}\dom f^*\neq\varnothing$ and $f^*$ is continuously differentiable on $\operatorname{int}\dom f^*$.
\end{theorem}

\begin{proof}
For arbitrary $x_0\in\dom f$ and nonzero $d\in\R^n$, consider Example 2.4.6. Strict convexity of $f$ implies that
\[
0<\frac{f(x_0 - td)-f(x_0)}{t}+\frac{f(x_0+td)-f(x_0)}{t}\qquad\text{for all }t>0,
\]
and this inequality extends to the suprema: $0<f'_ \infty(-d)+f'_\infty(d)$. Remembering that $f'_\infty=\sigma_{\dom f^*}$ (Proposition 1.2.2), this means
\[
\sigma_{\dom f^*}(d)+\sigma_{\dom f^*}(-d)>0,
\]
i.e.\ $\dom f^*$ has a positive breadth in every nonzero direction $d$: its interior is nonempty---Theorem C2.2.3(iii).

Now suppose that there is some $s\in\int\dom f^*$ such that $\partial f^*(s)$ contains two distinct points $x_1$ and $x_2$. Then $s\in\partial f(x_1)\cap\partial f(x_2)$; by convex combination of the relations
\[
f^*(s)+f(x_i)=\langle s,x_i\rangle\qquad\text{for }i=1,2
\]
we deduce, using Fenchel's inequality (1.1.3):
\[
f^*(s)+\sum_{i=1}^2\alpha_i f(x_i)=\langle s,\sum_{i=1}^2\alpha_i x_i\rangle\le f^*(s)+f\Big(\sum_{i=1}^2\alpha_i x_i\Big),
\]
which implies that $f$ is affine on $[x_1,x_2]$, a contradiction. In other words, $\partial f^*$ is single-valued on $\int\dom f^*$, and this means that $f^*$ is continuously differentiable there (Remark 6.2.6).
\end{proof}

\begin{theorem}\label{thm:FCA-chapE-4.1.2}
Let $f\in\operatorname{Conv}\mathbb{R}^n$ be differentiable on the set $\Omega:=\operatorname{int}\operatorname{dom}f$.  Then $f^*$ is strictly convex on each convex subset $C\subset\nabla f(\Omega)$.
\end{theorem}

\begin{proof}
Let $C$ be a convex set as stated.  Suppose that there are two distinct points $s_1$ and $s_2$ in $C$ such that $f^*$ is affine on the line-segment $[s_1,s_2]$.  Then, setting $s:=\tfrac12(s_1+s_2)\in C\subset\nabla f(\Omega)$, there is $x\in\Omega$ such that $\nabla f(x)=s$, i.e.\ $x\in\partial f^*(s)$.  Using the affine character of $f^*$, we have
\[
0=f(x)+f^*(s)-\langle s,x\rangle
=\tfrac12\sum_{i=1}^2\bigl[f(x)+f^*(s_i)-\langle s_i,x\rangle\bigr]
\]
and, in view of Fenchel's inequality (1.1.3), this implies that each term in the bracket is $0$: $x\in\partial f^*(s_1)\cap\partial f^*(s_2)$, i.e.\ $\partial f(x)$ contains the two points $s_1$ and $s_2$, a contradiction to the existence of $\nabla f(x)$.
\end{proof}

\begin{corollary}[Corollary 4.1.3]
Let $f:\mathbb{R}^n\to\mathbb{R}$ be strictly convex, differentiable and 1-coercive. Then
\begin{enumerate}
\item[(i)] $f^{*}$ is also finite-valued on $\mathbb{R}^n$, strictly convex, differentiable and 1-coercive;
\item[(ii)] the continuous mapping $\nabla f$ is one-to-one from $\mathbb{R}^n$ onto $\mathbb{R}^n$, and its inverse is continuous;
\end{enumerate}

\[
\text{(iii) }\; f^*(s)=\langle s,(\nabla f)^{-1}(s)\rangle - f\bigl((\nabla f)^{-1}(s)\bigr)\quad\text{for all }s\in\mathbb{R}^n.
\]
\end{corollary}

\subsection{Lipschitz continuity of the gradient mapping}

\begin{theorem}\label{thm:FCA-chapE-4.2.1}
Assume that $f:\mathbb{R}^n\to\mathbb{R}$ is strongly convex with modulus $c>0$ on $\mathbb{R}^n$: for all $(x_1,x_2)\in\mathbb{R}^n\times\mathbb{R}^n$ and $\alpha\in]0,1[$,
\[
f(\alpha x_1+(1-\alpha)x_2)\le \alpha f(x_1)+(1-\alpha)f(x_2)-\tfrac{1}{2}c\alpha(1-\alpha)\|x_1-x_2\|^2.
\tag{4.2.1}
\]
Then $\operatorname{dom} f^*=\mathbb{R}^n$ and $\nabla f^*$ is Lipschitzian with constant $1/c$ on $\mathbb{R}^n$:
\[
\|\nabla f^*(s_1)-\nabla f^*(s_2)\|\le\tfrac{1}{c}\|s_1-s_2\|\quad\text{for all }(s_1,s_2)\in\mathbb{R}^n\times\mathbb{R}^n.
\]
\end{theorem}

\begin{proof}
We use the various equivalent definitions of strong convexity (see Theorem D.6.1.2). Fix $x_0$ and $s_0\in\partial f(x_0)$: for all $0\neq d\in\mathbb{R}^n$ and $t\ge 0$
\[
f(x_0+td)\ge f(x_0)+t\langle s_0,d\rangle+\tfrac{1}{2}ct^2\|d\|^2,
\]
hence $f'_+(d)=\sigma_{\operatorname{dom} f^*}(d)=+\infty$, i.e.\ $\operatorname{dom} f^*=\mathbb{R}^n$. Also, $f$ is in particular strictly convex, so we know from Theorem 4.1.1 that $f^*$ is differentiable (on $\mathbb{R}^n$). Finally, strong convexity of $f$ can also be written $(s_1-s_2,x_1-x_2)\ge c\|x_1-x_2\|^2$, in which we have $s_i\in\partial f(x_i)$, i.e.\ $s_i=\nabla f^*(s_i)$, for $i=1,2$. The rest follows from the Cauchy--Schwarz inequality.
\end{proof}

\begin{theorem}\label{thm:FCA-chapE-4.2.2}
Let $f:\mathbb{R}^n\to\mathbb{R}$ be convex and have a gradient-mapping Lipschitzian with constant $L>0$ on $\mathbb{R}^n$: for all $(x_1,x_2)\in\mathbb{R}^n\times\mathbb{R}^n$,
\[
\|\nabla f(x_1)-\nabla f(x_2)\|\le L\|x_1-x_2\|.
\]
Then $f^*$ is strongly convex with modulus $1/L$ on each convex subset $C\subset\operatorname{dom}\partial f^*$.
In particular, there holds for all $(x_1,x_2)\in\mathbb{R}^n\times\mathbb{R}^n$
\[
\langle\nabla f(x_1)-\nabla f(x_2),\,x_1-x_2\rangle\ge \tfrac{1}{L}\|\nabla f(x_1)-\nabla f(x_2)\|^2.
\tag{4.2.2}
\]
\end{theorem}

\begin{proof}
Let $s_1$ and $s_2$ be arbitrary in $\dom\partial f^* \subset\dom f^*$; take $s$ and $s'$ on the segment $[s_1,s_2]$. To establish the strong convexity of $f^*$, we need to minorize the remainder term $f^*(s')-f^*(s)-\langle x,s'-s\rangle$, with $x\in\partial f^*(s)$. For this, we minorize $f^*(s')=\sup_y\{\langle s',y\rangle-f(y)\}$, i.e. we majorize $f(y)$:
\[
\begin{aligned}
f(y)&=f(x)+\langle\nabla f(x),\,y-x\rangle+\int_0^1\langle\nabla f(x+t(y-x))-\nabla f(x),\,y-x\rangle dt\\
&\le f(x)+\langle\nabla f(x),\,y-x\rangle+\tfrac12 L\|y-x\|^2\\
&= -f^*(s)+\langle s,y\rangle+\tfrac12 L\|y-x\|^2
\end{aligned}
\]
(we have used the property $\int_0^1 t\,dt=1/2$, as well as $x\in\partial f^*(s)$, i.e. $\nabla f(x)=s$). In summary, we have
\[
f^*(s')\ge f^*(s)+\sup_y\big[\,\langle s'-s,y\rangle-\tfrac12 L\|y-x\|^2\big].
\]
Observe that the last supremum is nothing but the value at $s'-s$ of the conjugate of $\tfrac12 L\|\,\cdot\,\|^2$. Using the calculus rule 1.3.1, we have therefore proved
\begin{equation}\label{thm:FCA-chapE-4.2.3}
f^*(s')\ge f^*(s)+\langle s'-s,x\rangle+\tfrac1{2L}\|s'-s\|^2
\end{equation}
for all $s,s'$ in $[s_1,s_2]$ and all $x\in\partial f^*(s)$. Replacing $s'$ in (4.2.3) by $s_1$ and by $s_2$, and setting $s=\alpha s_1+(1-\alpha)s_2$, the strong convexity (4.2.1) for $f^*$ is established by convex combination.

On the other hand, replacing $(s,s')$ by $(s_1,s_2)$ in (4.2.3):
\[
f^*(s_2)\ge f^*(s_1)+\langle s_2-s_1,x_1\rangle+\tfrac1{2L}\|s_2-s_1\|^2\quad\text{for all }x_1\in\partial f^*(s_1).
\]
Then, replacing $(s,s')$ by $(s_2,s_1)$ and summing: $\langle x_1-x_2,s_1-s_2\rangle\ge\tfrac1L\|s_1-s_2\|^2$. In view of the differentiability of $f$, this is just (4.2.2), which has to hold for all $(x_1,x_2)$ simply because $\Im\partial f^*=\dom\nabla f=\mathbb R^n$.
\end{proof}