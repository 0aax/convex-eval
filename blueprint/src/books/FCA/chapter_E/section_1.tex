\subsection{Definition and first examples}

\begin{definition}
The \emph{conjugate} of a function $f$ satisfying (1.1.1) is the function $f^{*}$ defined by
\[
\mathbb{R}^{n}\ni s\mapsto f^{*}(s):=\sup\{\langle s,x\rangle-f(x):x\in\operatorname{dom}f\}.
\tag{1.1.2}
\]
For simplicity, we may also let $x$ run over the whole space instead of $\operatorname{dom}f$.

The mapping $f\mapsto f^{*}$ will often be called the \emph{conjugacy} operation, or the Legendre–Fenchel transform.
\end{definition}

\begin{theorem}
\label{thm:FCA-chapE-1.1.2}
\lean{FCA_chap_E_1_1_2}
For $f$ satisfying (1.1.1), the conjugate $f^*$ is a closed convex function: $f^*\in\Conv\mathbb{R}^n$.
\end{theorem}

\begin{proof}
See Example B.2.1.3.
\end{proof}

\subsection{Interpretations}

\begin{proposition}
\label{prop:FCA-chapE-1.2.1}
\lean{FCA_chap_E_1_2_1}
There holds for all $x\in\mathbb{R}^n$
\[
f^*(s)=\sigma_{\mathrm{epi}\,f}(s,-1)=\sup\{\langle s,x\rangle - r : (x,r)\in\mathrm{epi}\,f\}.
\tag{1.2.1}
\]
It follows that the support function of $\mathrm{epi}\,f$ has the expression
\[
\sigma_{\mathrm{epi}\,f}(s,-u)=
\begin{cases}
u\,f^*\!\bigl(\tfrac{1}{u}s\bigr) & \text{if } u>0,\\[4pt]
\sigma_{\mathrm{epi}\,f}(s,0)=\sigma_{\mathrm{dom}\,f}(s) & \text{if } u=0,\\[4pt]
+\infty & \text{if } u<0.
\end{cases}
\tag{1.2.2}
\]
\end{proposition}

\begin{proof}
In (1.2.1), the right-most term can be written
\[
\sup_x\;\sup_{r\ge f(x)}\bigl[\langle s,x\rangle - r\bigr]
= \sup_x\bigl[\langle s,x\rangle - f(x)\bigr]
\]
and the first equality is established. As for (1.2.2), the case $u<0$ is trivial; when $u>0$, use the positive homogeneity of support functions to get
\[
\sigma_{\operatorname{epi} f}(s,-u)=u\sigma_{\operatorname{epi} f}\bigl(\tfrac{1}{u}s,-1\bigr)=u f^*\bigl(\tfrac{1}{u}s\bigr).
\]
Finally, for $u=0$, we have by definition
\[
\sigma_{\operatorname{epi} f}(s,0)=\sup\{ \langle s,x\rangle : (x,r)\in\operatorname{epi} f\text{ for some }r\in\mathbb{R}\},
\]
and we recognize $\sigma_{\operatorname{dom} f}(s)$.
\end{proof}

\begin{proposition}
\label{thm:FCA-chapE-1.2.2}
\lean{FCA_chap_E_1_2_2}
For $f\in\operatorname{Conv}\mathbb{R}^n$,
\[
\sigma_{\operatorname{dom} f}(s)=\sigma_{\operatorname{epi} f}(s,0)=(f^*)^\infty(s)\quad\text{for all }s\in\mathbb{R}^n.
\tag{1.2.3}
\]
\end{proposition}

\begin{proof}
Use direct calculations; or see Proposition B.2.2.2 and the calculations in Example B.3.2.3.
\end{proof}

\subsection{First properties}

\begin{proposition}
\label{prop:FCA-chapE-1.3.1}
\lean{FCA_chap_E_1_3_1_i,FCA_chap_E_1_3_1_ii,FCA_chap_E_1_3_1_iii,FCA_chap_E_1_3_1_iv,FCA_chap_E_1_3_1_v,FCA_chap_E_1_3_1_vi,FCA_chap_E_1_3_1_vii,FCA_chap_E_1_3_1_viii}
The functions $f$, $f_j$ appearing below are assumed to satisfy (1.1.1).
\begin{enumerate}
    \item[(i)] The conjugate of the function $g(x):=f(x)+r$ is $g^*(s)=f^*(s)-r$.
    \item[(ii)] With $t>0$, the conjugate of the function $g(x):=tf(x)$ is $g^*(s)=t f^*(s/t)$.
    \item[(iii)] With $t\neq0$, the conjugate of the function $g(x):=f(tx)$ is $g^*(s)=f^*(s/t)$.
    \item[(iv)] More generally: if $A$ is an invertible linear operator, $(f\circ A)^*=f^*\circ (A^{-1})^*$.
    \item[(v)] The conjugate of the function $g(x):=f(x-x_{0})$ is $g^{*}(s)=f^{*}(s)+\langle s,x_{0}\rangle$.
    \item[(vi)] The conjugate of the function $g(x):=f(x)+\langle s_{0},x\rangle$ is $g^{*}(s)=f^{*}(s-s_{0})$.
    \item[(vii)] If $f_{1}\le f_{2}$, then $f_{1}^{*}\ge f_{2}^{*}$.
    \item[(viii)] ``Convexity'' of the conjugation: if $\mathrm{dom}\,f_{1}\cap\mathrm{dom}\,f_{2}\neq\varnothing$ and $\alpha\in]0,1[$,
    \[
    [\alpha f_{1}+(1-\alpha)f_{2}]^{*}\le \alpha f_{1}^{*}+(1-\alpha)f_{2}^{*};
    \]
    \item[(ix)] The Legendre-Fenchel transform preserves decomposition: with
    \[
    \mathbb{R}^{n}:=\mathbb{R}^{n_{1}}\times\cdots\times\mathbb{R}^{n_{m}}\ni x\mapsto f(x):=\sum_{j=1}^{m}f_{j}(x_{j})
    \]
    and assuming that $\mathbb{R}^{n}$ has the scalar product of a product-space, there holds
    \[
    f^{*}(s_{1},\dots,s_{m})=\sum_{j=1}^{m}f_{j}^{*}(s_{j}) .
    \]
\end{enumerate}
\end{proposition}

\begin{proposition}
\label{prop:FCA-chapE-1.3.2}
\lean{FCA_chap_E_1_3_2}
Let $f$ satisfy (1.1.1), let $H$ be a subspace of $\mathbb{R}^n$, and call $p_H$ the operator of orthogonal projection onto $H$. Suppose that there is a point in $H$ where $f$ is finite. Then $f+p_{iH}$ satisfies (1.1.1) and its conjugate is
\[
(f+p_{iH})^*=(f\circ p_H)^*\circ p_H .
\tag{1.3.1}
\]
\end{proposition}

\begin{proof}
When $y$ describes $\mathbb{R}^n$, $p_H y$ describes $H$ so we can write, knowing that $p_H$ is symmetric:
\[
(f+p_{iH})^*(s):=\sup\{\langle s,x\rangle-f(x):x\in H\}
=\sup\{\langle s,p_H y\rangle-f(p_H y):y\in\mathbb{R}^n\}
=\sup\{\langle p_H s,y\rangle-f(p_H y):y\in\mathbb{R}^n\}.
\]
\end{proof}

\begin{proposition}
\label{prop:FCA-chapE-1.3.4}
\lean{FCA_chap_E_1_3_4}
For $f$ satisfying (1.1.1), let a subspace $V$ contain the subspace parallel to $\operatorname{aff\,dom}f$ and set $U:=V^{\perp}$. For any $z\in\operatorname{aff\,dom}f$ and any $s\in\mathbb{R}^{n}$ decomposed as $s=s_{U}+s_{V}$, there holds
\[
f^{*}(s)=\langle s_{U},z\rangle+f^{*}(s_{V}).
\]
\end{proposition}

\begin{proof}
In (1.1.2), the variable $x$ can range through $z+V\supset\operatorname{aff\,dom}f$:
\[
\begin{aligned}
f^{*}(s)&=\sup_{v\in V}[\langle s_{U}+s_{V},z+v\rangle-f(z+v)]\\
&=\langle s_{U},z\rangle+\sup_{v\in V}[\langle s_{V},z+v\rangle-f(z+v)]\\
&=\langle s_{U},z\rangle+f^{*}(s_{V}).
\end{aligned}
\]
\end{proof}

\begin{theorem}
\label{thm:FCA-chapE-1.3.5}
\lean{FCA_chap_E_1_3_5}
For $f$ satisfying (1.1.1), the function $f^{**}$ of (1.3.2) is the pointwise supremum of all the affine functions on $\mathbb{R}^n$ majorized by $f$.  In other words
\[
\operatorname{epi} f^{**}=\overline{\operatorname{co}}\bigl(\operatorname{epi} f\bigr).
\tag{1.3.3}
\]
\end{theorem}

\begin{proof}
Call $\Sigma\subset\mathbb{R}^n\times\mathbb{R}$ the set of pairs $(s,r)$ defining affine functions $\langle s,\cdot\rangle - r$ majorized by $f$:
\[
(s,r)\in\Sigma\iff f(x)\ge\langle s,x\rangle - r\quad\text{for all }x\in\mathbb{R}^n
\]
\[
\iff r\ge\sup\{\,\langle s,x\rangle - f(x):x\in\mathbb{R}^n\,\}
\]
\[
\iff r\ge f^*(s)\qquad(\text{and }s\in\operatorname{dom}f^*!).
\]

Then we obtain, for $x\in\mathbb{R}^n$,
\[
\sup_{(s,r)\in\Sigma}\{\langle s,x\rangle - r\}
= \sup\{\langle s,x\rangle - r : s\in\operatorname{dom}f^*,\ -r\le -f^*(s)\}
\]
\[
= \sup\{\langle s,x\rangle - f^*(s): s\in\operatorname{dom}f^*\}=f^{**}(x).
\]

Geometrically, the epigraphs of the affine functions associated with $(s,r)\in\Sigma$ are the (non-vertical) closed half-spaces containing $\operatorname{epi} f$.  From §B.2.5, the epigraph of their supremum is the closed convex hull of $\operatorname{epi} f$, and this proves (1.3.3).
\end{proof}

\begin{corollary}
\label{cor:FCA-chapE-1.3.6}
\lean{FCA_chap_E_1_3_6}
If $g$ is a function satisfying $\overline{\operatorname{co}}\,f \le g \le f$, then $g^* = f^*$. The function $f$ is equal to its biconjugate $f^{**}$ if and only if $f\in\overline{\operatorname{Conv}}\mathbb{R}^n$.
\end{corollary}

\begin{proof}
Immediate.
\end{proof}

\begin{definition}
A function $f$ satisfying (1.1.1) is said to be coercive [resp.\ 1-coercive] when
\[
\lim_{\|x\|\to+\infty} f(x)=+\infty
\qquad
\bigl[\text{resp. }\lim_{\|x\|\to+\infty}\frac{f(x)}{\|x\|}=+\infty\bigr].
\]
\end{definition}

\begin{proposition}
\label{prop:FCA-chapE-1.3.8}
\lean{FCA_chap_E_1_3_8}
If $f$ satisfying (1.1.1) is 1-coercive, then $f^*(s) < +\infty$ for all $s \in \mathbb{R}^n$.
\end{proposition}

\begin{proof}
Let $s$ be given. The 1-coercivity of $f$ implies the existence of a number $R$ such that $f(x) \ge \|s\|\,\|x\|$ (hence $\langle s,x\rangle - f(x) \le 0$) whenever $\|x\| \ge R$. As a result, we have in (1.1.2)
\[
\sup\{\langle s,x\rangle - \|s\|\,\|x\| : \|x\| \ge R\} \le 0.
\]
On the other hand, (1.1.1) implies an upper bound
\[
\sup \{\langle s,x\rangle - f(x) : \|x\| \le R\} \le M .
\]
Altogether, $f^*(s) \le \max\{0,M\}$. \qed
\end{proof}

\begin{proposition}
\label{prop:FCA-chapE-1.3.9}
\lean{FCA_chap_E_1_3_9}
For $f$ satisfying (1.1.1), the following holds:
\begin{enumerate}
\item[(i)] If $x_0\in\operatorname{int}\dom f$ then $f^*-\langle x_0,\cdot\rangle$ is $0$-coercive;
\item[(ii)] in particular, if $f$ is finite over $\mathbb R^n$, then $f^*$ is $1$-coercive.
\end{enumerate}
\end{proposition}

\begin{proof}
We know from (1.2.3) that $\sigma_{\dom f}=(f^*)^{\vee}_{\infty}$ so, using Theorem C.2.2.3(iii),
$x_0\in\operatorname{int}\dom f\subset\int(\codom f)$ implies $(f^*)^{\vee}_{\infty}(s)-\langle x_0,s\rangle>0$ for all $s\neq0$.
By virtue of Proposition B.3.2.4, this means exactly that $f^*-\langle x_0,\cdot\rangle$ has compact sublevel-sets; (i) is proved.

Then, as demonstrated in Definition B.3.2.5, $0$-coercivity of $f^*-\langle x_0,\cdot\rangle$ for all $x_0$ means $1$-coercivity of $f^*$.
\end{proof}

\subsection{Subdifferentials and extended-valued functions}

\begin{theorem}
\label{thm:FCA-chapE-1.4.1}
\lean{FCA_chap_E_1_4_1}
For $f$ satisfying (1.1.1) and $\partial f$ defined by (1.4.1), $s\in\partial f(x)$ if and only if
\[
f^*(s)+f(x)-\langle s,x\rangle=0\quad(\text{or }\le 0).
\tag{1.4.2}
\]
\end{theorem}

\begin{proof}
To say that $s$ lies in the set (1.4.1) is to say that
\[
\langle s,y\rangle-f(y)\le\langle s,x\rangle-f(x)\quad\text{for all }y\in\dom f,
\]
i.e.\ $f^*(s)\le\langle s,x\rangle-f(x)$; but this is indeed an equality, in view of Fenchel's inequality (1.1.3).
\end{proof}

\begin{theorem}
\label{thm:FCA-chapE-1.4.2}
\lean{FCA_chap_E_1_4_2}
Let $f\in\operatorname{Conv}\mathbb{R}^n$.  Then $\partial f(x)\neq\varnothing$ whenever $x\in\operatorname{ri}\dom f$.
\end{theorem}

\begin{proof}
This is Proposition B.1.2.1.
\end{proof}

\begin{proposition}
\label{prop:FCA-chapE-1.4.3}
\lean{FCA_chap_E_1_4_3}
For $f$ satisfying (1.1.1), the following properties hold:
\[
\partial f(x)\neq\varnothing \quad\Longrightarrow\quad (\overline{\operatorname{co}}\,f)(x)=f(x);\tag{1.4.3}
\]
\[
\overline{\operatorname{co}}\,f\le g\le f\ \text{ and }\ g(x)=f(x)
\quad\Longrightarrow\quad \partial g(x)=\partial f(x);\tag{1.4.4}
\]
\[
s\in\partial f(x)\quad\Longrightarrow\quad x\in\partial f^*(s).\tag{1.4.5}
\]
\end{proposition}

\begin{proof}
Let $s$ be a subgradient of $f$ at $x$. From the definition (1.4.1) itself, the function $y\mapsto \ell_s(y):=f(x)+\langle s,y-x\rangle$ is affine and minorizes $f$, hence $\ell_s\le \overline{\operatorname{co}}\,f\le f$; because $\ell_s(x)=f(x)$, this implies (1.4.3).

Now, $s\in\partial f(x)$ if and only if (1.4.2) holds. From our assumption in (1.4.4), $f^*=g^*=(\overline{\operatorname{co}}\,f)^*$ (Corollary 1.3.6) and $g(x)=f(x)$. Therefore
\[
s\in\partial f(x)\iff g^*(s)+g(x)-\langle s,x\rangle=0,
\]
which expresses exactly that $s\in\partial g(x)$; (1.4.4) is proved.

Finally, we know that $f^{**}=\overline{\operatorname{co}}\,f\le f$; so, when $s$ satisfies (1.4.2), we have
\[
f^*(s)+f^{**}(x)-\langle s,x\rangle
= f^*(s)+(\overline{\operatorname{co}}\,f)(x)-\langle s,x\rangle\le 0,
\]
which means $x\in\partial f^*(s)$: we have just proved (1.4.5).
\end{proof}

\begin{corollary}
\label{thm:FCA-chapE-1.4.4}
\lean{FCA_chap_E_1_4_4}
If $f\in\operatorname{Conv}\mathbb{R}^n$, the following equivalences hold:
\[
f(x)+f^*(s)-\langle s,x\rangle=0\ (\text{or }\le 0)
\quad\Longleftrightarrow\quad s\in\partial f(x)
\quad\Longleftrightarrow\quad x\in\partial f^*(s).
\]
\end{corollary}

\begin{proof}
This is a rewriting of Theorem 1.4.1, taking into account (1.4.5) and the symmetric role played by $f$ and $f^*$ when $f\in\operatorname{Conv}\mathbb{R}^n$.
\end{proof}