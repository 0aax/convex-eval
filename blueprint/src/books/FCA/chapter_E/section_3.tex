\subsection{The Cramer transform}

\subsection{The conjugate of a convex partially quadratic function}

\begin{proposition}
\label{prop:3.2.1}
The function $g$ of (3.2.1) has the conjugate
\[
g^{*}(s)=\begin{cases}
\tfrac{1}{2}\langle s, (p_{H}\circ B\circ p_{H})^{-}s\rangle & \text{if } s\in\operatorname{Im}B+H^{\perp},\\[4pt]
+\infty & \text{otherwise},
\end{cases}\tag{3.2.2}
\]
where $p_{H}$ is the operator of orthogonal projection onto $H$ and $(\cdot)^{-}$ is the Moore--Penrose pseudo-inverse.
\end{proposition}

\begin{proof}
Set $f:=\tfrac{1}{2}\langle B\cdot,\cdot\rangle$, so that $g=f+i_{H}$ and $g^{*}=(f\circ p_{H})^{*}\circ p_{H}$ (Proposition 1.3.2). Knowing that $(f\circ p_{H})(x)=\tfrac{1}{2}\langle (p_{H}\circ B\circ p_{H})x,x\rangle$, we obtain from Example 1.1.4 the conjugate $g^{*}(s)$ under the form
\[
(f\circ p_{H})^{*}(p_{H}s)=\begin{cases}
\tfrac{1}{2}\langle s,(p_{H}\circ B\circ p_{H})^{-}s\rangle & \text{if } p_{H}s\in\operatorname{Im}(p_{H}\circ B\circ p_{H}),\\[4pt]
+\infty & \text{otherwise}.
\end{cases}
\]

It could be checked directly that $\operatorname{Im}(p_{H}\circ B\circ p_{H})+H^{\perp}=\operatorname{Im}B+H^{\perp}$. A simpler argument, however, is obtained via Theorem 2.3.2, which can be applied since $\operatorname{dom}f=\mathbb{R}^{n}$. Thus,
\[
g^{*}(s)=(f^{*}\,\square\,i_{H^{\perp}})(s)=\min\{\tfrac{1}{2}\langle p,B^{-}p\rangle:\;p\in\operatorname{Im}B,\;s-p\in H^{\perp}\},
\]
which shows that $\operatorname{dom}g^{*}=\operatorname{Im}B+H^{\perp}$.
\end{proof}

\subsection{Polyhedral functions}

\begin{proposition}\label{prop:3.3.1}
At each $s\in\operatorname{co}\{s_1,\ldots,s_k\}=\operatorname{dom}f^*$, the conjugate of $f$ has the value ( $\Delta_k$ is the unit simplex)
\[
f^*(s)=\min\Big\{\sum_{i=1}^k \alpha_i b_i:\ \alpha\in\Delta_k,\ \sum_{i=1}^k \alpha_i s_i=s\Big\}.
\tag{3.3.2}
\]
\end{proposition}

\begin{proof}
Set $g_i(s):= \iota_{\{s_i\}}+b_i$ and
\[
g(s):=(\inf_i g_i)(s)=
\begin{cases}
b_i &\text{if } s=s_i\text{ for some } i=1,\ldots,k,\\[4pt]
+\infty &\text{otherwise}.
\end{cases}
\]
Apply Proposition B.2.5.4 to see that $\overline{\operatorname{co}}\,g=f^*$ of (3.3.2). The rest follows from Theorem 2.4.1 or 2.4.4, with a notational flip of $f$ and $g$. \qedhere
\end{proof}