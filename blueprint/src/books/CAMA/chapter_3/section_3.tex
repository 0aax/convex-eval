\subsection{The projection operator}

\begin{theorem}
\label{thm:3.1.1}
\lean{CAMA_chap_3_3_1_1}
A point $y_x\in C$ is the projection $p_C(x)$ if and only if
\[
\langle x-y_x,\;y-y_x\rangle\le 0\qquad\text{for all }y\in C .
\tag{3.1.3}
\]
\end{theorem}

\begin{proof}
Call $y_x$ the solution of (3.1.1); take $y$ arbitrary in $C$, so that $y_x+\alpha(y-y_x)\in C$ for any $\alpha\in ]0,1[$. Then we can write with the notation (3.1.2)
\[
f_x(y_x)\le f_x\big(y_x+\alpha(y-y_x)\big)=\tfrac12\|y_x-x+\alpha(y-y_x)\|^2.
\]
Developing the square, we obtain after simplification
\[
0\le \alpha\langle y_x-x,\;y-y_x\rangle+\tfrac12\alpha^2\|y-y_x\|^2.
\]
Divide by $\alpha\ (>0)$ and let $\alpha\downarrow0$ to obtain (3.1.3).

Conversely, suppose that $y_x\in C$ satisfies (3.1.3). If $y_x=x$, then $y_x$ certainly solves (3.1.1). If not, write for arbitrary $y\in C$:
\[
0\ge \langle x-y_x,\;y-y_x\rangle=(x-y_x,\;y-x+x-y_x)
\]
\[
=\|x-y_x\|^2+\langle x-y_x,\;y-x\rangle\ge\|x-y_x\|^2-\|x-y\|\;\|x-y_x\|,
\]
where the Cauchy-Schwarz inequality is used. Divide by $\|x-y_x\|>0$ to see that $y_x$ solves (3.1.1).
\end{proof}

\begin{proposition}
\label{prop:3.1.3}
\lean{CAMA_chap_3_3_1_3}
For all $(x_1,x_2)\in\mathbb{R}^n\times\mathbb{R}^n$, there holds
\[
\|p_C(x_1)-p_C(x_2)\|^2 \le \langle p_C(x_1)-p_C(x_2),\,x_1-x_2\rangle.
\]
\end{proposition}

\begin{proof}
Write (3.1.3) with $x=x_1$, $y=p_C(x_2)\in C$:
\[
\langle p_C(x_2)-p_C(x_1),\,x_1-p_C(x_1)\rangle \le 0;
\]
likewise,
\[
\langle p_C(x_1)-p_C(x_2),\,x_2-p_C(x_2)\rangle \le 0,
\]
and conclude by addition
\[
\langle p_C(x_1)-p_C(x_2),\,x_2-x_1 + p_C(x_1)-p_C(x_2)\rangle \le 0.
\]
\end{proof}

\subsection{Projection onto a closed convex cone}

\begin{definition}
\label{def:3.2.1}
Let $K$ be a convex cone, as defined in Example 1.1.4. The \emph{polar cone} of $K$ (called negative polar cone by some authors) is:
\[
K^{\circ} := \{ s \in \mathbb{R}^n : \langle s,x\rangle \le 0 \text{ for all } x \in K \}.
\]
\end{definition}

\begin{proposition}
\label{prop:3.2.3}
\lean{CAMA_chap_3_3_2_3}
Let $K$ be a closed convex cone.  Then $y_x = p_K(x)$ if and only if
\[
y_x \in K,\qquad x-y_x\in K^\circ,\qquad \langle x-y_x,y_x\rangle=0.
\tag{3.2.1}
\]
\end{proposition}

\begin{proof}
We know from Theorem 3.1.1 that $y_x = p_K(x)$ satisfies
\[
\langle x-y_x,y-y_x\rangle\le 0\qquad\text{for all }y\in K.
\tag{3.2.2}
\]
Taking $y=\alpha y_x$, with arbitrary $\alpha\ge 0$, this inequality implies
\[
(\alpha-1)\langle x-y_x,y_x\rangle\le 0\qquad\text{for all }\alpha\ge 0.
\]
Since $\alpha-1$ can have either sign, this implies $\langle x-y_x,y_x\rangle=0$ and (3.2.2) becomes
\[
\langle y,x-y_x\rangle\le 0\quad\text{for all }y\in K,\qquad\text{i.e.}\quad x-y_x\in K^\circ.
\]

Conversely, let $y_x$ satisfy (3.2.1).  For arbitrary $y\in K$, use the notation (3.1.2):
\[
f_x(y)=\tfrac12\|x-y_x+y_x-y\|^2\ge f_x(y_x)+\langle x-y_x,y_x-y\rangle;
\]
but (3.2.1) shows that
\[
\langle x-y_x,y_x-y\rangle=-\langle x-y_x,y\rangle\ge 0,
\]
hence $f_x(y)\ge f_x(y_x)$: $y_x$ solves (3.1.1).
\end{proof}

\begin{theorem}[J.-J. Moreau]
\label{thm:3.2.5}
\lean{CAMA_chap_3_3_2_5}
Let $K$ be a closed convex cone. For the three elements $x,x_1$ and $x_2$ in $\mathbb{R}^n$, the properties below are equivalent:
\begin{enumerate}
\item[(i)] $x=x_1+x_2$ with $x_1\in K$, $x_2\in K^\circ$ and $\langle x_1,x_2\rangle=0$;
\item[(ii)] $x_1=\mathrm{p}_K(x)$ and $x_2=\mathrm{p}_{K^\circ}(x)$.
\end{enumerate}
\end{theorem}

\begin{proof}
Straightforward, from (3.2.3) and the characterization (3.2.1) of $x_1=\mathrm{p}_K(x)$.
\end{proof}