\subsection{Convenient definitions of tangent cones}

\begin{definition}
\label{def:5.1.1}
Let $S\subset\mathbb{R}^n$ be nonempty. We say that $d\in\mathbb{R}^n$ is a direction \emph{tangent} to $S$ at $x\in S$ when there exists a sequence $\{x_k\}\subset S$ and a sequence $\{t_k\}$ such that, when $k\to+\infty$,
\[
x_k\to x,\qquad t_k\downarrow 0,\qquad \frac{x_k-x}{t_k}\to d.
\tag{5.1.1}
\]
The set of all such directions is called the \emph{tangent cone} (also called the contingent cone, or Bouligand's cone) to $S$ at $x\in S$, denoted by $T_S(x)$.
\end{definition}

\begin{proposition}
A direction $d$ is tangent to $S$ at $x\in S$ if and only if
\[
\exists (d_k)\to d,\ \exists (t_k)\downarrow 0\quad\text{such that}\quad x+t_kd_k\in S\ \text{for all }k.
\]
\end{proposition}

\begin{proposition}
\textit{The tangent cone is closed.}
\end{proposition}

\begin{proof}
Let $\{d_\ell\}\subset T_S(x)$ be converging to $d$; for each $\ell$ take sequences $\{x_{\ell,k}\}_k$ and $\{t_{\ell,k}\}_k$ associated with $d_\ell$ in the sense of Definition 5.1.1. Fix $\ell>0$: we can find $k_\ell$ such that
\[
\left\|\frac{x_{\ell,k_\ell}-x}{t_{\ell,k_\ell}}-d_\ell\right\|\le\frac{1}{\ell}.
\]
Letting $\ell\to\infty$, we then obtain the sequences $\{x_\ell,t_\ell\}_\ell$ and $\{t_\ell,k_\ell\}_\ell$ which define $d$ as an element of $T_S(x)$.
\end{proof}

\subsection{The tangent cones and normal cones to a convex set}

\begin{proposition}
\label{prop:5.2.1}
The tangent cone to $C$ at $x$ is the closure of the cone generated by $C-\{x\}$:
\begin{align}
T_C(x)
&=\overline{\operatorname{cone}}(C-x)=\overline{\operatorname{cl}}\mathbb{R}^+(C-x)\nonumber\\
&=\overline{\{d\in\mathbb{R}^n:\;d=\alpha(y-x),\;y\in C,\;\alpha\ge0\}}.
\tag{5.2.1}
\end{align}
\end{proposition}

\begin{proof}
We have just said that $C-\{x\}\subset T_C(x)$. Because $T_C(x)$ is a closed cone (Proposition 5.1.3), it immediately follows that $\overline{\mathbb{R}^+}(C-x)\subset T_C(x)$. Conversely, for $d\in T_C(x)$, take $\{x_k\}$ and $\{t_k\}$ as in the definition (5.1.1): the point $(x_k-x)/t_k$ is in $\mathbb{R}^+(C-x)$, hence its limit $d$ is in the closure of this latter set. \qedhere
\end{proof}

\begin{definition}
The direction $s\in\mathbb{R}^n$ is said \emph{normal} to $C$ at $x\in C$ when
\[
\langle s,\,y-x\rangle \le 0\quad\text{for all }y\in C .
\tag{5.2.2}
\]
The set of all such directions is called \emph{normal cone} to $C$ at $x$, denoted by $N_C(x)$.
\end{definition}

\begin{proposition}
\label{prop:5.2.4}
The normal cone is the polar of the tangent cone.
\end{proposition}

\begin{proof}
If $\langle s,d\rangle\le 0$ for all $d\in C-x$, the same holds for all $d\in\mathbb R^+(C-x)$, as well as for all $d$ in the closure $T_C(x)$ of the latter. Thus, $N_C(x)\subset[T_C(x)]^\circ$.

Conversely, take $s$ arbitrary in $[T_C(x)]^\circ$. The relation $\langle s,d\rangle\le 0$, which holds for all $d\in T_C(x)$, a fortiori holds for all $d\in C-x\subset T_C(x)$; this is just (5.2.2). \qedhere
\end{proof}

\begin{corollary}\label{cor:5.2.5}
The tangent cone is the polar of the normal cone:
\[
T_C(x)=\{d\in\mathbb{R}^n:\ \langle s,d\rangle\le 0\ \text{for all }s\in N_C(x)\}.
\]
\end{corollary}

\subsection{Some properties of tangent and normal cones}

\begin{proposition}{5.3.1}
Here, the $C$'s are nonempty closed convex sets.
\begin{enumerate}[(i)]
\item For $x\in C_1\cap C_2$, there holds
\[
T_{C_1\cap C_2}(x)\subset T_{C_1}(x)\cap T_{C_2}(x)
\qquad\text{and}\qquad
N_{C_1\cap C_2}(x)\supset N_{C_1}(x)+N_{C_2}(x).
\]

\item With $C_i\subset\mathbb R^{n_i}$, $i=1,2$ and $(x_1,x_2)\in C_1\times C_2$,
\[
T_{C_1\times C_2}(x_1,x_2)=T_{C_1}(x_1)\times T_{C_2}(x_2),
\qquad
N_{C_1\times C_2}(x_1,x_2)=N_{C_1}(x_1)\times N_{C_2}(x_2).
\]

\item With an affine mapping $A(x)=y_0+A_0x$ ($A_0$ linear) and $x\in C$,
\[
T_{A(C)}[A(x)]=\operatorname{cl}[A_0T_C(x)]
\qquad\text{and}\qquad
N_{A(C)}[A(x)]=A_0^{-*}[N_C(x)].
\]

\item In particular (start from (ii), (iii) and proceed as when proving (1.2.2)):
\[
T_{C_1+C_2}(x_1+x_2)=\operatorname{cl}[T_{C_1}(x_1)+T_{C_2}(x_2)],
\qquad
N_{C_1+C_2}(x_1+x_2)=N_{C_1}(x_1)\cap N_{C_2}(x_2).
\]
\end{enumerate}
\end{proposition}

\begin{proposition}
\label{prop:5.3.3}
For $x\in C$ and $s\in\mathbb{R}^n$, the following properties are equivalent:
\begin{enumerate}
\item[(i)] $s\in N_C(x)$;
\item[(ii)] $x$ is in the exposed face $F_C(s)$: $\langle s,x\rangle=\max_{y\in C}\langle s,y\rangle$;
\item[(iii)] $x=p_C(x+s)$.
\end{enumerate}
\end{proposition}

\begin{proof}
Nothing really new: everything comes from the definitions of normal cones, supporting hyperplanes, exposed faces, and the characteristic property (3.1.3) of the projection operator.
\end{proof}

\begin{proposition}
For given $x\in C$ and $d\in\mathbb{R}^n$, there holds
\[
\lim_{t\downarrow0}\frac{\operatorname{P}_{C}(x+td)-x}{t}=\operatorname{P}_{T_C(x)}(d).
\tag{5.3.3}
\]
\end{proposition}

\begin{proof}
HINT. Start from the characterization (3.1.3) of a projection, to observe that the difference quotient $[\operatorname{P}_C(x+td)-x]/t$ is the projection of $d$ onto $(C-x)/t$. Then let $t\downarrow0$; the result comes as well with the help of (5.1.4) and Remark 5.2.2.
\end{proof}